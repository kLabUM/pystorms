\documentclass{article}

\usepackage{amsmath}

\title{A primer of the control of urban water systems}
\author{Abhiram Mullapudi}

\begin{document}
\maketitle

\section{Urban water systems}

% What are urban water systems?
urban drinage systems are networks of links and nodes that convey water in urban environments.

classically these systems are developed as built as a static infrastructure or with someform of localized control.
but given that changes that we are seeing in the environment around us, our infrastructure is having trouble catching up with it.

sensing and communication enable us to monitor the system in near-real time and we can operate them to use these systems to fullest of their ability. 

these networks can be classified into three:
\begin{itemize}
	\item storm water networks
	\item drinking water networks
	\item sanitary systems
\end{itemize}
through these systems can all be controlled, in this session we focus on the storm water systems.

\section{Control of water systems}
Control of water networks is by no means new, its has been going on since the beginning of time.

one thing that comes to mind is the resorvior optimization stuff and industrial engineering. 

now one can ask, why are we not directly using these methods?

well, for a bunch of reasons. 

1. It is not that they are not usable. But it is more like they have limitations.
At the core, we will be using these methods, but we will be adapting how we approach control to leverage these methodologies. 

The challenging part of it how do we take these methods and apply it to our problems.

Model predictive control, you have to develop a linear model and then use it for controlling things.
there are couple of amazing works that are out there that do this. 

\section{Types of Controllers}

On a broad scale, control of water system can be classified into the following types:

\begin{itemize}
	\item rule based control approaches
	\item model predictive control approaches
	\item hrustic based methods
\end{itemize}

this is by no means an final classification or a complete one. But for the purposes of our workshop, I believe that it is helpful. 
\end{document}
